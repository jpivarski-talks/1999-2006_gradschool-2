\documentclass[12pt]{article}
\pagestyle{empty}

\oddsidemargin  -0.5 cm
\evensidemargin 0.0 cm
\textwidth      6.5in
\headheight     0.0in
\topmargin      -1 cm
\textheight=9.0in

\setlength{\parindent}{0 cm}

\begin{document}

\begin{center}
  \sc Curriculum Vitae \\
  Jim Pivarski
\end{center}

\subsection*{Contact Information}

\begin{center}
  \begin{tabular}{p{9 cm} p{1 cm} p{4 cm}}
    \begin{minipage}{\linewidth}
      Gaslight Village Apt.\ \#4A \\
      Ithaca, NY, 14850
    \end{minipage} & &
    \begin{minipage}{\linewidth}
      (607) 257 3510 \\
      {\tt mccann@watson.org}
    \end{minipage} \smallskip \\
    \begin{minipage}{\linewidth}
      Wilson Laboratory of Particle Physics ({\sc LEPP}) \\
      Cornell University, Ithaca, NY, 14850
    \end{minipage} & &
    \begin{minipage}{\linewidth}
      (607) 254 2774 \\
      {\tt mccann@lns.cornell.edu}
    \end{minipage}
  \end{tabular}
\end{center}

\subsection*{Education/Prior Research Experience}

\begin{tabular}{p{0.18\linewidth} p{0.8\linewidth}}
  \begin{minipage}[t]{\linewidth}
    \begin{center}
      \it 1999 - present
    \end{center}
  \end{minipage} & \begin{minipage}[t]{\linewidth}
    Ph.D.\ program in Physics at Cornell University, Ithaca, NY.
  \end{minipage} \smallskip \\

  \begin{minipage}[t]{\linewidth}
    \begin{center}
      \it 1995 - 1999
    \end{center}
  \end{minipage} & \begin{minipage}[t]{\linewidth}
    B.S.\ in Physics with a minor in Mathematical Sciences \\
    from Carnegie Mellon University, Pittsburgh, PA.
  \end{minipage} \smallskip \\

  \begin{minipage}[t]{\linewidth}
    \begin{center}
      \it 1999
    \end{center}
  \end{minipage} & \begin{minipage}[t]{\linewidth}
    Informal work on QCD Sum Rule calculations of glueball content of the
    $f_0$ triplet under Leonard Kisslinger at Carnegie Mellon.
  \end{minipage} \smallskip \\

  \begin{minipage}[t]{\linewidth}
    \begin{center}
      \it 1998
    \end{center}
  \end{minipage} & \begin{minipage}[t]{\linewidth}
    R.E.U.\ studying a prototype \v{C}erenkov detector \\
    at Thomas Jefferson National Laboratory, Newport News, VA.
  \end{minipage} \smallskip \\

  \begin{minipage}[t]{\linewidth}
    \begin{center}
      \it 1997
    \end{center}
  \end{minipage} & \begin{minipage}[t]{\linewidth}
    R.E.U.\ studying E831 trigger efficiency \\
    at Fermi National Laboratory, Batavia, IL.
  \end{minipage} \smallskip \\

  \begin{minipage}[t]{\linewidth}
    \begin{center}
      \it \small May 2002 \\
      and \\
      May 2003
    \end{center}
  \end{minipage} & \begin{minipage}[t]{\linewidth}
    Presented work on $\Gamma_{\Upsilon(1,2,3S) \to e^+e^-}$ (described below) \\
    at the American Physics Society.
  \end{minipage}

\end{tabular}

\subsection*{Current Work/Research}

{\sc \bf CLEO} {\bf Collaboration} under advisor Ritchie Patterson

\vspace{-0.1 cm}
\begin{quote}
I am currently measuring the leptonic partial width
($\Gamma_{e^+e^-}$) of $\Upsilon(1S)$, $\Upsilon(2S)$, and
$\Upsilon(3S)$ from scans of the hadronic cross-section lineshapes of
these three resonances by {\sc Cleo-III} ($e^+e^- \to \Upsilon \to
\mbox{hadrons}$).  With 600-840 $pb^{-1}$ per resonance, this
measurement is limited only by systematic errors, which I am carefully
studying with the reasonable goal of reducing to $\mathcal
O($1-2\%$)$.

These three measurements will be compared to unquenched-lattice QCD
calculations of the same quantities, for a high-precision comparison
between theory and experiment.

Also, I am or have been responsible for the following:
\end{quote}
\begin{itemize}
  
  \item Alignment of tracking detectors in track reconstruction and
  assignment of fit weights

  \item Maintainance of track-fitting code (finding and fixing bugs)

  \item Supervising an undergraduate student who is refurbishing a
  cosmic ray demonstration

\end{itemize}

\end{document}

