\documentclass[12pt]{article}

\oddsidemargin  -0.5 cm
\evensidemargin 0.0 cm
\textwidth      6.5in
\headheight     0.0in
\topmargin      -1 cm
\textheight=9.0in

\begin{document}

\begin{flushright}
Jim Pivarski's Notes for Physics and Geometry (\today)
\end{flushright}

\section{Maxwell's GUT}

Three empirically-discovered laws of electrodynamics were known in
Maxwell's time:
\begin{eqnarray}
  \vec{E} &=& \displaystyle \frac{1}{4\pi\epsilon_0} \int \frac{\rho}{r^2} \hat{r} \, dr \mbox{\hspace{0.9 cm} Coulumb's Law} \\
  \vec{B} &=& \displaystyle \frac{\mu_0}{4\pi} \int \frac{\vec{J}\times \hat{r}}{r^2} \, dr \mbox{\hspace{0.8 cm} Biot-Savart Law} \\
  \displaystyle \oint_\Omega \vec{E} \cdot d\vec{\ell} &=& \displaystyle -\frac{d}{dt} \int_\Omega \vec{B} \cdot d\vec{a} \mbox{\hspace{1 cm} Faraday's Law} \\
  \vec{F} &=& \displaystyle q(\vec{E} + \vec{v} \times \vec{B}) \mbox{\hspace{1 cm} Lorentz Force Law}
\end{eqnarray}
(The last is a definition of $\vec{E}$ and $\vec{B}$ in terms of
mechanical force.)  $\vec{E}$ is the electric field, $\vec{B}$ is the
magnetic field, $\rho$ is the density of electric charge, $\vec{J}$ is
the density of electric charge current, $q$ is a test charge with
velocity $\vec{v}$, $dr$ is a differential over all space, $d\vec{a}$
and $d\ell$ are differentials over the area and bounary of a surface
$\Omega$.  These equations can be written in differential form like
this:
\begin{eqnarray}
  \nabla \cdot \vec{E} &=& \displaystyle \frac{1}{\epsilon_0} \rho \mbox{\hspace{1 cm} Gauss's Law} \\
  \nabla \times \vec{B} &=& \mu_0 \vec{J} \mbox{\hspace{0.9 cm} Amp\`ere's Law} \\
  \nabla \cdot \vec{B} &=& 0 \\
  \displaystyle \nabla \times \vec{E} + \frac{\partial \vec{B}}{\partial t} &=& 0 \mbox{\hspace{1.4 cm} Faraday's Law}
\end{eqnarray}

Maxwell noticed a theoretical problem with this theory: if you take
the divergence of Amp\`ere's Law,
\begin{equation}
  \nabla \cdot (\nabla \times \vec{B}) = \mu_0 \nabla \cdot \vec{J}
\end{equation}
the right hand side is not in general zero.  Thought of as a fluid,
charge density is compressible, and if charge is flowing into a volume
element (the plate of a capacitor is charging up), it will increase
the charge density in that volume element.
\begin{equation}
  -\nabla \cdot \vec{J} = \frac{\partial \rho}{\partial t}
\end{equation}
Combining this with Gauss's Law, we see that Amp\`ere's Law can be
generalized by adding a $\mu_0\epsilon_0 \partial\vec{E}/\partial t$
term.  Now the full Maxwell Equations read

\begin{center}
  \fbox{\begin{minipage}{\linewidth}
    \vspace{-0.5 cm}
    \begin{eqnarray}
      \nabla \cdot \vec{E} &=& \displaystyle \frac{1}{\epsilon_0} \rho \label{me1} \\
      \displaystyle \nabla \times \vec{B} - \mu_0\epsilon_0 \frac{\partial \vec{E}}{\partial t} &=& \mu_0 \vec{J} \label{me2} \\
      \nabla \cdot \vec{B} &=& 0 \label{me3} \\
      \displaystyle \nabla \times \vec{E} + \frac{\partial \vec{B}}{\partial t} &=& 0 \label{me4}
    \end{eqnarray}
  \end{minipage}}
\end{center}

\subsection{Symmetry Between Electric and Magnetic Fields}

These equations are very symmetric with the exception of their right
hand sides.  If $\rho$ and $\vec{J}$ are zero, one could swap the
roles of $\vec{E}$ and $\vec{B}$, providing one appropriately handles
the constants $\epsilon_0$ and $\mu_0$.  I'm going to handle them by
choosing some unit system in which they are both equal to 1.

One can actually rotate the definitions
\begin{equation}
  \left( \begin{array}{c} \vec{E}' \\ \vec{B}' \end{array} \right) =
  \left( \begin{array}{c c} \cos\theta & \sin\theta \\ -\sin\theta & \cos\theta \end{array} \right)
  \left( \begin{array}{c} \vec{E} \\ \vec{B} \end{array} \right)
\end{equation}
where $\vec{E}'$ and $\vec{B}'$ are how electric and magnetic fields
are defined on some alien planet.

If we {\it leave in} the right hand side and prime it (because it's
alien), Maxwell's equations become
\begin{eqnarray}
  \nabla \cdot \vec{E}' &=& \displaystyle \rho' \\
  \displaystyle \nabla \times \vec{B}' - \frac{\partial \vec{E}'}{\partial t} &=& \vec{J}' \\
  \nabla \cdot \vec{B}' &=& 0 \\
  \displaystyle \nabla \times \vec{E}' + \frac{\partial \vec{B}'}{\partial t} &=& 0
\end{eqnarray}
which reduces to
\begin{eqnarray}
  \nabla \cdot \vec{E} &=& \displaystyle \cos\theta \rho' \\
  \displaystyle \nabla \times \vec{B} - \frac{\partial \vec{E}}{\partial t} &=& \cos\theta \vec{J}' \\
  \nabla \cdot \vec{B} &=& \sin\theta \rho' \\
  \displaystyle \nabla \times \vec{E} + \frac{\partial \vec{B}}{\partial t} &=& -\sin\theta \vec{J}'
\end{eqnarray}
The sources have also undergone a rotation: $\vec{E}$ and $\vec{B}$
form a two-dimensional space of possible definitions, and the sources
are rotated along with the fields.  But the sources observed in nature
all lie along a line in this space that no rotation can smear, and so
Earthlings have adopted a convention of labelling the fields with
sources {\it electric} and the fields without sources {\it magnetic.}

Deviation from the thin charged line, a particle observed to have
$\nabla \cdot \vec{B} \ne 0$ according to our definition of $\vec{B}$,
would be called a magnetic monopole.  P.\ A.\ M.\ Dirac argued, within
the context of quantum mechanics, that the existence of a single
magnetic monopole anywhere in the universe will suffice to quantize
electric charge everywhere.  Despite the really good reason for there
to be magnetic monopoles, 147 experiments between 1951 and 2004 saw
only four events, and none of them were deemed significant given the
parameters of the experiments.  But then, we only need one.

Since I intend to avoid quantum mechanics in this lecture, I won't
present the argument.  (Maybe Matt will.)

\subsection{Maxwell's Equations, Relativised.}

Classical mechanics need to be modified to become relativistic, but
Maxwell's Equations are already relativistic.  In fact, it was taking
Maxwell's Equations seriously that led A.\ Einstein to relativity
(``On the Electrodynamics of Moving Bodies'' in 1905).  I'll just
write them in four-vector notation and move on.

\subsubsection{Cheat-Sheet for Einstein Notation}

(\ldots because I would probably forget and drop a $\sum$ if I tried
to do it right.)
\begin{itemize}

  \item We express vector equations by showing what is done to each
  component: $x^\mu = 0$ means the vector $x$ equals $(0,0,0,0)$.

  \item A sub/superscript appearing on both sides of the equation has
  an implied ``$\forall$'': to say $x^\mu = y^\mu$ implies $\forall
  \mu$.

  \item A sub/superscript appearing mulitply on one side of an
  equation (or only in one term) has an implied ``$\sum$'': to say
  $x^\mu y_\mu = 0$ implies $\displaystyle \sum_\mu$.

  \item What is the meaning of upper and lower indices?  In special
  relativity (which is all {\it I} will be doing), nothing.  But in
  curved spaces, ntuples with a raised index are contravariant vectors
  and ntuples with a lowered index are covariant one-forms.  I'd like
  to understand this distinction better myself.  Only upper and lower
  indices can be contracted ($\sum_\mu x^\mu y_\mu$ is good, $\sum_\mu
  x^\mu y^\mu$ is bad), and a lower index can be converted into an
  upper index by contracting with the funkymetric $g^{\mu\nu}$: $x^\mu =
  g^{\mu\nu} x_\nu$ where in special relativity
\begin{equation}
  g^{\mu\nu} = \left(\begin{array}{c c c c}
-1 & 0 & 0 & 0 \\
0 & 1 & 0 & 0 \\
0 & 0 & 1 & 0 \\
0 & 0 & 0 & 1 \end{array}\right)
\end{equation}
  For the remainder of {\it my} talk, this whole bullet point will not
  matter.

  \item If you are keeping track of indices anyway, we should know
  that $\partial/\partial x^\mu$ transforms as a one-form with a {\it
  lowered} index.  This is why I write all space-time derivatives with
  a single symbol $\partial_\mu$.

\end{itemize}

Non-relativistic physics usually involves interactions between a
scalar quantity and a vecotr quantity that mean nearly the same thing.
Time is a coordinate and space is a coordinate; energy is a conserved
quantity of motion and momentum is a conserved quantity of motion;
charge density is a measure of electricity and currernt density is a
measure of electricity.  These can be combined into four-vectors by
making the scalar piece the time (zeroth) component:
\begin{eqnarray}
  x^\mu &=& (t,x,y,z)^\mu \\
  p^\mu &=& (E,p_x,p_y,p_z)^\mu \\
  j^\mu &=& (\rho, J_x, J_y, J_z)^\mu
\end{eqnarray}

Maxwell's Equations involve the interactions of two three-vectors---
six independent components.  An antisymmetric four-tensor has six
independent components.  Ergo (Q.\ E.\ D.) Maxwell's Equations can be
written in terms of an antisymmetric {\it field strength tensor.}
\begin{equation}
  F^{\mu\nu} = \left(\begin{array}{c c c c}
0 & E_x & E_y & E_z \\
-E_x & 0 & B_z & -B_y \\
-E_y & -B_z & 0 & B_x \\
-E_z & B_y & -B_x & 0
\end{array}\right)^{\mu\nu}
\end{equation}
It's done like this:
\begin{equation}
  \partial_\mu F^{\mu\nu} = \left(\begin{array}{c}
\displaystyle 0 + \frac{\partial E_x}{\partial x} + \frac{\partial E_y}{\partial y} + \frac{\partial E_z}{\partial z} \\
\displaystyle -\frac{\partial E_x}{\partial t} + 0 + \frac{\partial B_z}{\partial y} - \frac{\partial B_y}{\partial z} \\
\displaystyle -\frac{\partial E_y}{\partial t} - \frac{\partial B_z}{\partial x} + 0 + \frac{\partial B_x}{\partial z} \\
\displaystyle -\frac{\partial E_z}{\partial t} + \frac{\partial B_y}{\partial x} - \frac{\partial B_x}{\partial y} + 0
  \end{array}\right)^{\mu\nu}
= \left(\begin{array}{c} \rho \\ J_x \\ J_y \\ J_z \end{array}\right)^\nu = j^\nu
\end{equation}

Thus the first two Maxwell Equations (a scalar \ref{me1} and
three-vector \ref{me2}) can be rolled up into one relativistic
equation without changing any numerical predictions.  (Classical
mechanics changes $E = p^2/2m$ into $E^2 = p^2 + m^2$.)  For the other
two, dual rotate $\vec{E}$ and $\vec{B}$ 90 degrees ($\vec{E} \to
\vec{B}$ and $\vec{B} \to -\vec{E}$) to get
\begin{equation}
  G^{\mu\nu} = \left(\begin{array}{c c c c}
0 & B_x & B_y & B_z \\
-B_x & 0 & -E_z & E_y \\
-B_y & E_z & 0 & -E_x \\
-B_z & -E_y & E_x & 0
\end{array}\right)^{\mu\nu}
\end{equation}
Now we can write Maxwell's Equations as
\begin{equation}
  \partial_\mu F^{\mu\nu} = j^\nu \mbox{ and } \partial_\mu G^{\mu\nu} = 0
\end{equation}
which is what really should be on the T-shirts.

\section{The Potential and Gauge Freedom}

In electrostatics, force-problems could be turned into energy-problems
by replacing $\vec{E}$ with a potential $V$.  In three dimensions,
\begin{equation}
  \vec{E} = -\nabla V
\end{equation}
Because $V$ was obtained by integration, it is undetermined up to a
scalar constant $V_0$.  Only potential {\it differences} lead to
observable physics.

In full electrodynamics, the potential needs three more dimensions:
\begin{eqnarray}
  \vec{E} &=& -\nabla V - \frac{\partial \vec{A}}{\partial t} \\
  \vec{B} &=& \nabla \times \vec{A}
\end{eqnarray}
As usual, one scalar $+$ one three-vector makes a four-vector.
\begin{equation}
  A^\mu = (V, A_x, A_y, A_z)^\mu
\end{equation}

$\vec{E}$ and $\vec{B}$ (which are directly measured by the forces
they inflict on a test charge) are unchanged by translations of
$A^\mu$ by the gradient of any scalar field $\chi$.
\begin{equation}
  (A_\mu)' = A_\mu + \partial_\mu \chi
\end{equation}
This is, in three-vector land,
\begin{equation}
  V' = V - \frac{\partial \chi}{\partial t} \mbox{ and }
  \vec{A}' = \vec{A} + \nabla \chi
\end{equation}

This freedom to choose and constrain $A^\mu$ (called {\it gauge
freedom}) can be used to simplify calculations.  Also, it is deeply
responsible for the coupling of charged matter to electromagnetic
fields.  Here's a gauge I'll use in the next section, called the {\it
Lorentz} or {\it radiation gauge:}
\begin{equation}
  \partial_\mu A^\mu = 0
  \label{gauge}
\end{equation}
This doesn't use up all the gauge freedom--- we can still translate
\begin{equation}
  (A_\mu)' = A_\mu + \partial_\mu \chi \mbox{ as long as } \partial_\mu \partial^\mu \chi = 0
  \label{aslongas}
\end{equation}

\section{Applications and Interpretation}

We can write the stress field tensor in terms of $A^\mu$
\begin{equation}
  F^{\mu\nu} = \partial^\mu A^\nu - \partial^nu A^\mu
\end{equation}
and therefore Maxwell's Equations are
\begin{equation}
  \partial_\mu F^{\mu\nu} = \partial_\mu \partial^\mu A^\nu - \partial^\nu \partial_\mu A^\mu = j^\nu
\end{equation}
In Lorentz gauge, $\partial_\mu A^\mu$ is zero, so the second term is
zero.  The wave equation for $A^\mu$ is therefore
\begin{equation}
  \partial_\mu \partial^\mu A^\nu = j^\nu
  \label{inhomogeneous}
\end{equation}
This is a simple wave equation in each of the four components $\nu$:
\begin{equation}
  -\frac{\partial^2 A^\nu}{\partial t^2} + \nabla^2 A^\nu = j^\nu
\end{equation}

As an illustration, imagine $j^z = \sin t$ at one point on an $XY$
plane: $A^z$ waves are generated in the plane, propogating outward
from this point.  This is a cross-section of an infinite radio
antenna: waves of $A^\mu$ are generated by oscillating electric
current.

Without a source ($j^\nu = 0$), the wavefunction $A^\nu$ has solutions
\begin{equation}
  A^\nu = \varepsilon^\nu(p) e^{-i p_{\mu'} x^{\mu'}}
  \label{solution}
\end{equation}
where the energy/momentum of the wave is the four-vector $p_{\mu'}$,
and $\varepsilon^\nu$ is a four-component spinor for a given $p$.

\subsection{A Word About Quantum Mechanics}

``Wavefunction'' is the right word: the homogeneous equation
\begin{equation}
  \frac{\partial^2 A^\nu}{\partial t^2} - \nabla^2 A^\nu = 0
\end{equation}
is a Klein-Gordon equation for a boson with zero mass.  It is related to
the non-relativistic Schr\"odinger Equation
\begin{equation}
  i\frac{\partial \psi}{\partial t} + \frac{1}{2m} \nabla^2 \psi = 0
\end{equation}
(which is $E - p^2/2m = 0$ with $E \to i \partial/\partial t$ and $\vec{p}
\to i\nabla$) by using the relativistic expression $E^2 - p^2 - m^2 =
0$.  Each $A^\nu$ takes the place of $\psi$.

My treatment for this talk will be classical in the sense that
$A^\nu(x)$ is a real (or complex) number.  In quantum mechanics (or,
correctly speaking, quantum field theory), $A^\nu$ is replaced with an
abstract operator that obeys a non-trivial commutation relation.  This
commutation relation restricts the spectrum of solutions to
non-negative integers, so the $N$th excited state above the vacuum is
said to contain $N$ photons.  In general, a physical state may be a
superposition of the $N$-photon states.  The light incident on a
photomultiplier tube, for instance, is usually in a Poisson
distribution on $N$ with a mean of about 50 photons.  The difference
between a quantum mechanical description and a classical description,
therefore, is usually something like the difference between a Poission
distribution with a mean of 49.3 and a single-valued real number with
value of 49.3.

\section{Properties of the Classical Photofield}

Substitute the solution (\ref{solution}) into the homogeneous wave equation
(equation \ref{inhomogeneous} with $j^\nu = 0$), and we get
\begin{eqnarray}
  \partial_\mu \partial^\mu A^\nu &=& \partial_\mu (-i p_\mu) \varepsilon^\nu(p) e^{-i p_{\mu'} x^{\mu'}} \\
  &=& (-p_\mu p^\mu) \varepsilon^\nu(p) e^{-i p_{\mu'} x^{\mu'}} \\
  0 &=& p_\mu p^\mu
\end{eqnarray}
The length of the energy/momentum vector is always zero.  In this
non-positive-definite funkymetric, a line with zero length is any that
satisfies
\begin{equation}
  \Delta t^2 = \Delta x^2 + \Delta y^2 + \Delta z^2
\end{equation}
The set of all such lines from a given point is a light cone.

In terms of three-vectors and scalars, this means that $E = |p|^2$ for
all photons.

At first glance, \ref{solution} has four components, but physical
light has only two independent polarizations.  The reduction in
components comes from the gauge freedom.

Take $\chi$ in \ref{aslongas} to be
\begin{equation}
  \chi = ie^{-i p_{\mu'} x^{\mu'}}
\end{equation}
(This satisfies $\partial_\mu \partial^\mu \chi = 0$ for the same
reason as $\partial_\mu \partial^\mu A^\nu = 0$.)

The gauge translation in $A^\nu$ is
\begin{equation}
  (A^\nu)' = A^\nu + \partial^\nu \chi = A^\nu + p^\nu e^{-i p_{\mu'} x^{\mu'}}
\end{equation}
so the physical state represented by \ref{solution} is also
represented by 
\begin{eqnarray}
  A^\nu &=& \varepsilon^\nu(p) e^{-i p_{\mu'} x^{\mu'}} + p^\nu e^{-i p_{\mu'} x^{\mu'}} \\
  &=& (\varepsilon^\nu(p) + p^\nu) e^{-i p_{\mu'} x^{\mu'}}
\end{eqnarray}
The spinor $\varepsilon^\nu$ is physically the same as the spinor
$\varepsilon^\nu + p^\nu$!  It can be scaled arbitrarily along its
direction of propogation!  Let's use this freedom to adopt a
convention for polarization in the ``time'' direction:
\begin{equation}
  \varepsilon^0 = 0
\end{equation}

The rest of our gauge freedom is the Lorentz condition $\partial_\nu
A^\nu = 0$.  Let's insert the solution (\ref{solution}) into this
equation:
\begin{eqnarray}
  0 = \partial_\nu A^\nu &=& (-i p_nu) \varepsilon^\nu(p) e^{-i p_{\mu'} x^{\mu'}} \\
  0 &=& p_\nu \varepsilon^\nu(p)
\end{eqnarray}
Because of our convention, this is $\vec{p} \cdot \vec{\varepsilon} =
0$, which is to say that the components of the photofield spinor, the
polarization of light, is perpendicular to the direction of
propogation.

\section{Connecting the Electrofield to the Photofield}

To talk about electrodynamics more abstractly, I need to introduce the
Lagrangian.  In classical mechanics, the Lagrangian is a scalar
function over parameters relevant to the system being modelled, from
which equations of motion can be derived by finding the
minimum-Lagrangian path through parameter space.  The equations of
motion are the Euler-Lagrange Equations
\begin{equation}
  \frac{d}{dt}\left(\frac{\partial \mathcal{L}}{\partial (\partial q_i/\partial t)}
  \right) - \frac{\partial \mathcal{L}}{\partial q_i} = 0
\end{equation}
where $q_i$ are the parameters of the system, and they all depend on
the one coordinate $t$.

For field theories, the variables of the Lagrangian $q_i(t)$ become
$A^\nu(t,x,y,z)$.  In this generalization, the time derivative becomes
$\partial_\mu$:
\begin{equation}
  \partial_\mu \left(\frac{\partial \mathcal{L}}{\partial (\partial_\mu A^\nu)}
  \right) - \frac{\partial \mathcal{L}}{\partial A^\nu} = 0
\end{equation}

The Lagrangian for electromagnetism, what should really go on the
T-shirts, is
\begin{eqnarray}
  \mathcal{L}_{EM} &=& -\frac{1}{4} F_{\mu'\nu'} F^{\mu'\nu'} - j^{\mu'} A_{\mu'} \\
  &=& -\frac{1}{4} (\partial_{\mu'}A_{\nu'} - \partial_{\nu'}A_{\mu'})(\partial^{\mu'}A^{\nu'} - \partial^{\nu'}A^{\mu'}) - j^{\mu'} A_{\mu'}
\end{eqnarray}
 derivatives of the Lagrangian are
\begin{eqnarray}
  \frac{\partial \mathcal{L}_{EM}}{\partial (\partial_\mu A^\nu)} &=& -F^{\mu\nu} \mbox{ and} \\
  \frac{\partial \mathcal{L}_{EM}}{\partial A^\nu} &=& -j^\nu
\end{eqnarray}
recovering Maxwell's Equations as the equations of motion:
\begin{equation}
  \partial_\mu F^{\mu\nu} = j^\nu
\end{equation}

\subsection{Electrofield}

In the same way that we consider a classical photon field by refusing
the turn complex numbers into operators, we can consider the wave
equation for electrons classical.  The Dirac Equation for fermions can
be derived from the following Lagrangian
\begin{equation}
  \mathcal{L}_D = i \psi^\dagger \gamma^0 \gamma_\mu \partial^\mu \psi
  - m \psi^\dagger \gamma^0 \psi
\end{equation}
where
\begin{equation}
  \gamma^0 = \left(\begin{array}{c c c c}
1 & & & \\
 & 1 & & \\
 & & -1 & \\
 & & & -1 \end{array}\right) \mbox{\hspace{0.3 cm}}
  \gamma^1 = \left(\begin{array}{c c c c}
 & & 0 & -i \\
 & & -i & 0\\
0 & i & & \\
i & 0 & & \end{array}\right)
\end{equation}
\begin{equation}
  \gamma^2 = \left(\begin{array}{c c c c}
 & & 0 & -1 \\
 & & 1 & 0 \\
0 & 1 & & \\
-1 & 0& & \end{array}\right) \mbox{\hspace{0.3 cm}}
  \gamma^3 = \left(\begin{array}{c c c c}
 & & -i & 0 \\
 & & 0 & i\\
i & 0 & & \\
0 & -i & & \end{array}\right)
\end{equation}
and $\psi$ is a vector with four complex components.  (This
four-dimensional space is not space-time, it's the minimum number of
components necessary to construct a wave equation that is first-order
in time and space derivatives.  Coincidentally, that mathematical
necessity provides a natural origin for the existence of
two-dimensional spinors $\times$ the two components of particle and
antiparticle.)

\subsection{Noether's Theorm (a.k.a.\ The Deepest Theorem in Physics)}

\subsection{Local U(1) Invariance of $\mathcal{L}_D$ Needs Coupling to the Photofield}

\subsection{The Complete Electromagnetic Lagrangian}

\begin{center}
  \fbox{\begin{minipage}{\linewidth}
    \begin{equation}
      \mathcal{L} = i \psi^\dagger \gamma^0 \gamma_\mu \partial^\mu \psi
      - m \psi^\dagger \gamma^0 \psi
      - \frac{1}{4} F_{\mu'\nu'} F^{\mu'\nu'} + e \psi^\dagger \gamma^0 \gamma^\mu A_\mu
    \end{equation}
  \end{minipage}}
\end{center}

\end{document}
