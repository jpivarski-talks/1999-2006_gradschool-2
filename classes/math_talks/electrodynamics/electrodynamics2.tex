\documentclass[12pt]{article}
\usepackage{amssymb,graphicx}

\oddsidemargin  -0.5 cm
\evensidemargin 0.0 cm
\textwidth      6.5in
\headheight     0.0in
\topmargin      -1 cm
\textheight=9.0in

\begin{document}

\begin{flushright}
Jim Pivarski's Notes for Physics and Geometry (\today)
\end{flushright}

\section{Lagrangians are Important!}

First principles of classical mechanics (no more information than
Newton's Laws, but a simpler formulation)
\begin{equation}
  \mathcal{L} : \{q_i, \dot{q_i}\} \to \mathbb{R} \mbox{ where }
  q_i : \mathbb{R} \mbox{ (time) } \to \mathbb{R} \mbox{ (coordinate value) and }
  \dot{q_i} = \frac{dq}{dt}
\end{equation}
Hamilton's Principle (1835): of all possible paths a system defined by
$q_i$ and $\dot{q_i}$ can take between $t_1$ and $t_2$, the path a
physical system actually does take minimizes
\begin{equation}
  \int_{t_1}^{t_2} \mathcal{L} \, dt \mbox{ (called the ``action'').}
\end{equation}
The minimization of paths yields the equations of motion
\begin{equation}
  \frac{d}{dt}\left(\mbox{``}\frac{\partial \mathcal{L}}{\partial \dot{q_i}}\mbox{''}
  \right) - \mbox{``}\frac{\partial \mathcal{L}}{\partial q_i}\mbox{''} = 0
\end{equation}

Symmetries of the Lagrangian generate conserved quantities.  For
instance, time translation symmetry generates conservation of energy:
\begin{eqnarray}
  \frac{d}{dt}\mathcal{L} &=& \sum_i \mbox{``} \frac{\partial \mathcal{L}}{\partial q_i} \mbox{''} \dot{q_i} +
  \sum_i \mbox{``} \frac{\partial \mathcal{L}}{\partial \dot{q_i}} \mbox{''} \frac{d\dot{q_i}}{dt} 
  \mbox{ ($+$ no explicit $\displaystyle \frac{\partial\mathcal{L}}{\partial t}$)} \\
  &=& \sum_i \left( \frac{d}{dt} \mbox{``} \frac{\partial \mathcal{L}}{\partial \dot{q_i}} \mbox{''} \right) \dot{q_i} +
  \sum_i \mbox{``} \frac{\partial \mathcal{L}}{\partial \dot{q_i}} \mbox{''} \left(\frac{d}{dt}\dot{q_i}\right) \\
  &=& \sum_i \frac{d}{dt} \left( \mbox{``} \frac{\partial \mathcal{L}}{\partial \dot{q_i}} \mbox{''} \dot{q_i} \right) \\
  0 &=& \frac{d}{dt} \left(\mathcal{L} - \sum_i\mbox{``} \frac{\partial \mathcal{L}}{\partial \dot{q_i}} \mbox{''} \dot{q_i} \right)
\end{eqnarray}
where the parenthesized expression in the last row is the Hamiltonian,
the energy of the system.

For field theories,
\begin{eqnarray}
  \mathcal{L} &:& \{\psi_i, \partial_\mu \psi_i, \psi_i^\dagger, \partial_\mu \psi_i^\dagger\} \to \mathbb{R} \\
  \mbox{ where } \psi_i &:& \mathbb{R}^4 \mbox{ (space-time) } \to \mathbb{R}, \mathbb{C}, \mbox{ or non-commutative operator (QM)}
\end{eqnarray}
where the equations of motion are now
\begin{equation}
  \sum_i \left( \partial_\mu \left(\mbox{``}\frac{\partial \mathcal{L}}{\partial (\partial_\mu \psi_i)} \mbox{''}
  \right) - \mbox{``} \frac{\partial \mathcal{L}}{\partial \psi_i} \mbox{''} +
  \partial_\mu \left(\mbox{``} \frac{\partial \mathcal{L}}{\partial (\partial_\mu \psi_i^\dagger)} \mbox{''}
  \right) - \mbox{``} \frac{\partial \mathcal{L}}{\partial \psi_i^\dagger} \mbox{''} \right) = 0
\end{equation}

\section{Phase Symmetry Generates Charge Conservation}

Take the Dirac Lagrangian
\begin{equation}
  \mathcal{L}_D = i \psi^\dagger \gamma^0 \gamma_\mu \partial^\mu \psi - m \psi^\dagger \gamma^0 \psi
\end{equation}
which describes the kinetic ($\psi^\dagger \partial^\mu \psi$) and
potential ($m \psi^\dagger \psi$) energy of electrons and positrons.
Here, $\psi$ is complex, but the Lagrangian is insensitive to the
phase of $\psi$.  If we took
\begin{eqnarray}
  \psi &\to& \psi' = e^{i \alpha} \psi \mbox{,} \label{globalphase} \\
  \mathcal{L} &\to& \mathcal{L}' = \mathcal{L} \mbox{.}
\end{eqnarray}

Let $\mathcal{L}$ be any Lagrangian insensitive to the phase of
$\psi$.  Suppose we perturbed $\psi$ by a small phase change $\delta
\psi = i \alpha \psi$.  The change in the Lagrangian would be
\begin{eqnarray}
  0 &=& \delta \mathcal{L} = \mbox{``}\frac{\partial\mathcal{L}}{\partial \psi}\mbox{''} \delta \psi +
  \mbox{``}\frac{\partial\mathcal{L}}{\partial (\partial_\mu \psi)}\mbox{''} \delta (\partial_\mu \psi) +
  \mbox{``}\frac{\partial\mathcal{L}}{\partial \psi^\dagger}\mbox{''} \delta \psi^\dagger +
  \mbox{``}\frac{\partial\mathcal{L}}{\partial (\partial_\mu \psi^\dagger)}\mbox{''} \delta (\partial_\mu \psi^\dagger) \\
  &=& \mbox{``}\frac{\partial\mathcal{L}}{\partial \psi}\mbox{''} (i \alpha \psi) +
  \mbox{``}\frac{\partial\mathcal{L}}{\partial (\partial_\mu \psi)}\mbox{''} (i \alpha \partial_\mu \psi) +
  \mbox{ same for $\psi^\dagger$} \\
  &=& i \alpha \left(\mbox{``}\frac{\partial\mathcal{L}}{\partial \psi}\mbox{''}\psi +
  \mbox{``}\frac{\partial\mathcal{L}}{\partial (\partial_\mu \psi)}\mbox{''} (\partial_\mu \psi) -
  \partial_\mu \left(\mbox{``}\frac{\partial\mathcal{L}}{\partial (\partial_\mu \psi)}\mbox{''} \psi \right) \right)
  + i \alpha \partial_\mu \left(\mbox{``}\frac{\partial\mathcal{L}}{\partial (\partial_\mu \psi)}\mbox{''} \psi \right)
  + ^\dagger \\
  0 &=& i \alpha \left( \underbrace{\mbox{``}\frac{\partial\mathcal{L}}{\partial \psi}\mbox{''} -
  \partial_\mu \left(\mbox{``}\frac{\partial\mathcal{L}}{\partial (\partial_\mu \psi)}\mbox{''}\right)} \right) \psi
  + i \alpha \partial_\mu \left( \mbox{``}\frac{\partial\mathcal{L}}{\partial (\partial_\mu \psi)}\mbox{''} \psi \right)
  + \mbox{ same for $\psi^\dagger$}
\end{eqnarray}
where the underbraced part is left-hand side of the Euler-Lagrange
equation, and therefore zero.  That leaves us with a conserved
quantity
\begin{equation}
  \partial_\mu \left( \mbox{``}\frac{\partial\mathcal{L}}{\partial (\partial_\mu \psi)}\mbox{''} \psi -
  \mbox{``}\frac{\partial\mathcal{L}}{\partial (\partial_\mu \psi^\dagger)}\mbox{''} \psi^\dagger \right) = 0
\end{equation}
Call it $j^\mu$, so that
\begin{equation}
  \partial_\mu j^\mu = 0 \mbox{ or } -\nabla \cdot \vec{J} = \frac{\partial \rho}{\partial t}
\end{equation}

For the Dirac Lagrangian, this conserved quantity is
\begin{equation}
  j_D^\mu = i \psi^\dagger \gamma^0 \gamma_\mu \psi
\end{equation}
This suggests joining the Dirac Lagrangian to the Electromagnetic Lagrangian
\begin{equation}
  \mathcal{L}_{EM} = -\frac{1}{4}F_{\mu\nu}F^{\mu\nu} - j_{EM}^\mu A_\mu
\end{equation}
by identifying $j_{EM} = -i e j_D$.  ($e$ is an arbitrary constant, which
in nature turns out to be $\sqrt{4 \pi \alpha_{EM}} =$ 0.302822\ldots)

\section{From Global to Local Phase Invariance}

Instead of saying that there is an unmeasurable universal phase (but
phase differences in $\psi$ can be measured from between any two
arbitrarily distant points), suppose that we require the phase of
$\psi$ to be unmeasurable at every point.  In other words, instead of
requiring the Lagrangian to be invariant under \ref{globalphase},
require it to be invariant under
\begin{equation}
  \psi(x) \to \psi'(x) = e^{i \alpha(x)} \psi(x) \mbox{ (where $x$ is a space-time point)}
\end{equation}

The Dirac Lagrangian (alone) is not invariant under this more
restrictive condition.  Because
\begin{equation}
  \partial_\mu (\psi') = i(\partial_\mu \alpha) e^{i\alpha} \psi + e^{i\alpha} (\partial_\mu \psi) \mbox{,}
\end{equation}
\begin{eqnarray}
  \mathcal{L}_D' &=& i(e^{-i\alpha})\psi^\dagger \gamma^0 \gamma^\mu
  \left(e^{i\alpha} (\partial_\mu \psi) + i (\partial_\mu \alpha) e^{i\alpha} \psi \right) - m \psi^\dagger \gamma^0 \psi \\
  &=& i \psi^\dagger \gamma^0 \gamma^\mu (\partial_\mu \psi) - m \psi^\dagger \gamma^0 \psi -
  \psi^\dagger \gamma^0 \gamma^\mu \psi (\partial_\mu \alpha) \\
  &=& \mathcal{L}_D -   \psi^\dagger \gamma^0 \gamma^\mu \psi (\partial_\mu \alpha) \label{needtocancel}
\end{eqnarray}
This can be fixed up by putting in that $e \psi^\dagger \gamma^0
\gamma_\mu \psi A^\mu$ term where
\begin{equation}
  A^\mu \to (A^\mu)' = A^\mu + \frac{1}{e}\partial_\mu \alpha \label{gaugesymmetryofa}
\end{equation}
(LINKS PHASE SYMMETRY WITH GAUGE SYMMETRY!)  This new term
transforms as
\begin{eqnarray}
  (\mbox{``}e\mbox{''} \psi^\dagger \gamma^0 \gamma^\mu \psi A_\mu)' &=&
  \mbox{``}e\mbox{''} (e^{-i\alpha}) \psi^\dagger \gamma^0 \gamma^\mu (e^{i\alpha}) \psi
  (A_\mu + \frac{1}{\mbox{``}e\mbox{''}} \partial_\mu \alpha ) \\
  &=& \mbox{``}e\mbox{''} \psi^\dagger \gamma^0 \gamma^\mu \psi A_\mu + \psi^\dagger \gamma^0 \gamma^\mu \psi (\partial_\mu \alpha)
\end{eqnarray}
The second term is the one we need to cancel the extra one in
\ref{needtocancel}, making the Lagrangian
\begin{equation}
  \mathcal{L}_D + e \psi^\dagger \gamma^0 \gamma_\mu \psi A^\mu
\end{equation}
invariant under local phase/gauge symmetry.  Since we've just provided
$A^\mu$ as an energy sink for electrons and positrons, we should also
include the photofield's kinetic energy.
\begin{center}
  \fbox{\begin{minipage}{\linewidth}
    \begin{equation}
      \mathcal{L}_{\mbox{\scriptsize all of E\&M}} = i \psi^\dagger \gamma^0 \gamma_\mu \partial^\mu \psi
      - m \psi^\dagger \gamma^0 \psi
      - \frac{1}{4} F_{\mu'\nu'} F^{\mu'\nu'} + e \psi^\dagger \gamma^0 \gamma^\mu A_\mu
    \end{equation}
  \end{minipage}}
\end{center}

While we're including all possibilities, why not add a mass term for
the photon?  It might have mass\ldots after all, electrons and
positrons have mass.  A mass term for a vector particle would look
like this:
\begin{equation}
  \frac{1}{2} m^2 A_\mu A^\mu
\end{equation}
But that's not locally gauge invariant.
\begin{equation}
  (\frac{1}{2} m^2 A_\mu A^\mu)' = \frac{1}{2} m^2 A_\mu A^\mu +
  \frac{1}{2} \frac{m^2}{e}(A_\mu \partial^\mu \alpha + A^\mu \partial_\mu \alpha) +
  \frac{1}{2} \frac{m^2}{e}(\partial_\mu \alpha)(\partial^\mu \alpha)
\end{equation}

\section{Photon(-ish Things) With Mass}

Weak Nuclear Force is just like Electromagnetism except that it's
generated by a local $SU(2)$ symmetry, rather than $U(1)$.  (Instead
of rotating $\psi(x)$, a complex number, by $e^{i \alpha(x)}$, rotate
$\Psi(x)$, a two-dimensional complex vector, by $e^{i \vec{\alpha}(x)
\cdot \vec{T}}$ where $\vec{T}$ is a vector of Pauli spinors.)  Other
than that, the Weak Force Lagrangian is just like the Electromagnetic
Lagrangian--- Weak currents were generated from local $SU(2)$
symmetry--- but the vector fields $(W^+)_\mu$, $(W^-)_\mu$, and
$(Z^0)_\mu$ have mass!

\begin{center}
  \begin{tabular}{p{0.5\linewidth} p{0.5\linewidth}}
    \begin{minipage}{\linewidth}
      \includegraphics[width=\linewidth]{wzlep.ps}
    \end{minipage} &
    \begin{minipage}{\linewidth}
      Precision Weak Force physics at LEP:
      \begin{itemize}
        \item 18 million $Z$ bosons with a mass of 91.1882 GeV (1989 - 2000)

        \item 80 thousand $W^+W^-$ pairs with a threshold of
              2$\times$80.419 GeV (1996-2000)

	\item (Contribution to $e^+e^- \to$ hadrons centered at zero
              comes from the zero-mass photon.)
      \end{itemize}
    \end{minipage}
  \end{tabular}
\end{center}

How can local $SU(2)$ symmetry be justified in a theory where the
gauge fields have mass?

\section{General Discussion of Mass Terms}

A mass term is a negative term in the Lagrangian that grows
quadratically with the amplitude of a field.  It is an energy
cost for having too much field: a (moving) massless field only costs
as much as the kinetic energy it carries, but massive fields also
carry a fee for having the field at all.  In quantum mechanics, where
fields are quantized into a number of particles, the number of
particles is proportional to the field amplitude squared.

The $-m\psi^\dagger \gamma^0 \psi$ term in the Dirac Lagrangian is a
mass term because it is proportional to $-m|\psi|^2$.  A vector field
like a photon would have a $(1/2) m^2 A_\mu A^\mu$ mass term.  (The
sign and units being different has something to do with the Dirac equation
being first order.)  For the Weak Nuclear Force, we need something
like $(1/2) m^2 A_\mu A^\mu$ for $W+$, $W-$, and $Z^0$, but we can't
put it in by hand, because that would break local gauge symmetry.

So far, we've discussed ``spin-1/2'' Dirac fermions (electrons and
positrons) and ``spin-1'' vector fields (photons), but there are other
possibilities.  An interesting one is a ``spin-0'' scalar field, which
can have a Lagrangian like this:
\begin{equation}
  \label{higgslagrangian}
  \mathcal{L}_H = \partial^\mu \phi^* \partial_\mu \phi - \mu^2
  \phi^*\phi - \lambda (\phi^* \phi)^2 \mbox{ where $\phi(x)$ is a
  complex number.}
\end{equation}
The $-\mu^2 \phi^*\phi$ term is the scalar field mass and the $\lambda
(\phi^* \phi)^2$ term allows the field to interact with itself. What
would happen if we made $\mu^2$ negative?

Normally, you can't have a negative mass term because the Lagrangian
would be unbounded from below, and it would be advantageous to
continually create field, but the scalar has a negative definite
$-\lambda (\phi^* \phi)^2$ term, so free production of field is
bounded.  Rather than having a quadratic form, the mass-potential will
have two minima at $\pm\sqrt{-\mu^2/\lambda}$.
\begin{center}
  \begin{tabular}{p{0.4\linewidth} p{0.4\linewidth}}
    \includegraphics[width=\linewidth]{quadratic.eps} &
    \includegraphics[width=\linewidth]{mexicanhat.eps} \\
    \vspace{-0.5 cm} \begin{center} A normal mass-potential\end{center} &
    \vspace{-0.5 cm} \begin{center}Scalar field with negative $\mu^2$\end{center}
  \end{tabular}
\end{center}
Actually, since $\phi$ is complex, the minimum potential energy
solutions lie along a circle in the Argand plane with radius
\begin{equation}
  v = \sqrt{-\mu^2/\lambda}
\end{equation}
($v$ is for vacuum expectation value, the
typical low-energy value of $\phi$.)  Because the Lagrangian has
neither one local minimum nor is that minimum unique, nature has to
choose a ground state for the scalar field to fall into.  This is
called spontaneous symmetry breaking.

\section{The Higgs Mechanism}

Including a scalar field can give a vector field like $A^\mu$ an
effective mass.  In the following Lagrangian, a scalar field interacts
with a vector field:
\begin{equation}
  \mathcal{L}_{HA} = (\partial^\mu + ieA^\mu) \phi^* (\partial_\mu - ieA_\mu) \phi
  - \mu^2 \phi^*\phi - \lambda (\phi^*\phi)^2 - \frac{1}{4} F_{\mu\nu}F^{\mu\nu}
\end{equation}
once again, $\mu^2$ is negative, $\phi(x)$ is complex, and $A^\mu$
transforms as in \ref{gaugesymmetryofa}.  The $\pm ie A^\mu$ terms in
the derivatives of $\phi$ guarantee local $U(1)$ symmetry of
$\mathcal{L}$.

The complex scalar field $\phi$ can be re-written in terms of two real
fields $h$ and $\theta$ like this:
\begin{equation}
  \phi(x) = \sqrt{\frac{1}{2}} (v + h(x)) e^{i\theta(x)/v}
\end{equation}
I'd like to choose the gauge where $\alpha(x) = \theta(x) / v$, so
that $A^\mu$ transforms as
\begin{equation}
  A^\mu \to (A^\mu)' = A^\mu + \frac{1}{ev} \partial^\mu \theta
\end{equation}
The first term in the Lagrangian becomes
\begin{eqnarray}
  & & \frac{1}{2}(\partial^\mu + ieA^\mu + \frac{i}{v} \partial^\mu \theta)(v+h)e^{-i\theta/v}
  (\partial_\mu - ieA_\mu - \frac{i}{v} \partial_\mu \theta) (v + h) e^{i\theta/v} \\
  &=& \frac{1}{2}(\partial^\mu h e^{-i\theta/v} + (v+h)\frac{-i}{v} \partial^\mu \theta e^{-i\theta/v} + ieA^\mu(v+h)e^{-i\theta/v} + \frac{i}{v}(\partial^\mu \theta)(v+h)e^{-i\theta/v}) \\
  & & \times (\partial_\mu h e^{i\theta/v} + (v+h)\frac{i}{v} \partial_\mu \theta e^{i\theta/v} - ieA_\mu(v+h)e^{i\theta/v} - \frac{i}{v}(\partial_\mu \theta)(v+h)e^{i\theta/v}) \\
  &=& \frac{1}{2}(\partial^\mu h + ieA^\mu (v+h)) (\partial_\mu h - ieA_\mu(v+h)) \\
  &=& \frac{1}{2}(\partial^\mu h) (\partial_\mu h) + \frac{1}{2} e^2 A^\mu A_\mu (v + h)^2
\end{eqnarray}
The next two terms become
\begin{equation}
  \frac{1}{4} v^4 \lambda - v^2\lambda h^2 - v \lambda h^3 - \frac{1}{4} \lambda h^4
\end{equation}
The last term, $-\frac{1}{4} F_{\mu\nu}F^{\mu\nu}$, is unchanged.

The full Lagrangian is therefore
\begin{equation}
  \mathcal{L}_{HA}' = \frac{1}{2}(\partial^\mu h) (\partial_\mu h) + \fbox{$\displaystyle \frac{1}{2}e^2v^2 A^\mu A_\mu$} + e^2 v h A^\mu A_\mu + \frac{1}{2}e^2 h^2 A^\mu A_\mu
  - \fbox{$v^2 \lambda h^2$} - v \lambda h^3 - \frac{1}{4} \lambda h^4 + \frac{1}{4} v^4 \lambda^2
\end{equation}

The second term is an effective mass for the $A^\mu$ vector field: a mass of
\begin{equation}
  m_A = ev
\end{equation}
is generated by the scalar field $\phi$.  The $h$ field also has a mass,
\begin{equation}
  m_h = v \sqrt{2\lambda}
\end{equation}
The other terms are higher-order interactions of $h$ with itself and
$h$ and $A^\mu$ with each other.

In the Standard Model, $\phi$ couples to the $SU(2)$ part of the
Lagrangian (algebraically more complicated but the same spirit as the
above), generating the masses of $W^+$, $W^-$, and $Z^0$, but not the
photon $A$.  The effective field $h$ has a mass, which was
conveniently too large to generate in colliders (at the time the Higgs
mechanism was proposed), but there are good reasons to expect the LHC
at CERN to discover it.

\end{document}
