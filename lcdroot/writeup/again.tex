\documentclass[12pt]{article}
\usepackage[pdftex]{graphicx}

\oddsidemargin  -0.5 cm
\evensidemargin 0.0 cm
\textwidth      6.5in
\headheight     0.0in
\topmargin      -1 cm
\textheight=9.0in

\begin{document}

\section{Neutralinos to Two Leptons}

\begin{figure}[t]
  \begin{center}
    Electrons right-polarized 95\%, positrons unpolarized
    \begin{tabular}{p{0.49\linewidth} p{0.49\linewidth}}
      \begin{minipage}{\linewidth} \includegraphics[width=\linewidth]{again1.pdf} \end{minipage} &
      \begin{minipage}{\linewidth} \includegraphics[width=\linewidth]{again2.pdf} \end{minipage}
    \end{tabular}

    \caption{Analysis of neutralinos from 250 fb$^{-1}$ of $e^+e^-$
    collisions: electrons 95\% right-polarized, positrons unpolarized.
    Left-top: Missing energy for events with zero jets and two tracks
    identified as leptons.  Left-bottom: Invariant mass of the two
    leptons with $>$ 350 GeV missing energy.  Right: Invariant mass
    with opposite-flavor leptons subtracted and fit to a threshold
    function.  (All Z-pair events are in the high bin at 91 GeV.)}

    \label{jimptwoleptons}
  \end{center}
\end{figure}

\begin{figure}[t]
  \begin{center}
    Electrons left-polarized 95\%, positrons unpolarized
    \begin{tabular}{p{0.49\linewidth} p{0.49\linewidth}}
      \begin{minipage}{\linewidth} \includegraphics[width=\linewidth]{again1l.pdf} \end{minipage} &
      \begin{minipage}{\linewidth} \includegraphics[width=\linewidth]{again2l.pdf} \end{minipage}
    \end{tabular}

    \caption{Analysis of neutralinos from 250 fb$^{-1}$ of $e^+e^-$
    collisions: electrons 95\% left-polarized, positrons unpolarized.
    Left-top: Missing energy for events with zero jets and two tracks
    identified as leptons.  Left-bottom: Cosine of the polar angle of
    the positive track from the positron beam-line minus that of the
    negative track, with $>$ 350 GeV missing energy.  The cut at zero
    divides the signal symmetrically and avoids the region where
    $W^\pm$ is correlated in direction with the incident $e^\pm$.
    Right: Invariant mass with opposite-flavor leptons subtracted and
    fit to a threshold function, with missing energy and angle cuts
    applied.  (All Z-pair events are in the high bin at 91 GeV.)}

    \label{jimpleftleptons}
  \end{center}
\end{figure}

\begin{center}
  \begin{tabular}{l l l}
    $e^+e^- \to \tilde{\chi}^0_1$ & $\tilde{\chi}^0_3$ & \\
    & $\tilde{\chi}^0_3 \to \tilde{\chi}^0_1$ & $Z^*$ \\
    & & $Z^* \to \ell^+ \ell^-$
  \end{tabular}

\bigskip
and

\bigskip
  \begin{tabular}{l l l l l}
    $e^+e^- \to $ & $\tilde{\chi}^0_2$ & & $\tilde{\chi}^0_3$ & \\
    & $\tilde{\chi}^0_2 \to \tilde{\chi}^0_1$ & $Z^*$ & $\tilde{\chi}^0_3 \to \tilde{\chi}^0_1$ & $Z^*$ \\
    & & $Z^* \to \nu \bar{\nu}$ & & $Z^* \to \ell^+ \ell^-$
  \end{tabular}
\end{center}

\bigskip

Both neutralino modes, $\tilde{\chi}^0_1\tilde{\chi}^0_3$ and
$\tilde{\chi}^0_2\tilde{\chi}^0_3$, can be identified by selecting
events with two oppositely-charged tracks and missing energy.  This
lets the $\tilde{\chi}^0_3$ decay into a $Z^0 \to \ell^+\ell^-$ (two
electrons or muons) and $\tilde{\chi}^0_1$ (missing energy).  (The
prompt $\tilde{\chi}^0_1$ also contributes to missing energy.)  While
the neutralino signals can be cleanly distinguished from Standard
Model and chargino backgrounds, it is difficult to separate the two
signals from each other because $\tilde{\chi}^0_2$ can decay invisibly
into $\tilde{\chi}^0_1$ with a $Z^0 \to \nu \bar{\nu}$.  Consequently,
the two-lepton final state can only be used to measure a weighted sum
of the two modes, to be disambiguated later by the two jet, two lepton
measurement of $\tilde{\chi}^0_2\tilde{\chi}^0_3$ alone (see section
\ref{X}).  One measurement that is actually aided by the overlapping
modes is the mass difference between $\tilde{\chi}^0_3$ and
$\tilde{\chi}^0_1$; both neutralinos have an upper limit in invariant
mass at this mass difference.

The most effective means of background suppression comes from the
availability of polarized beams: if only the electron beam is 95\%
right-polarized, W-pair events in the signal region (missing energy)
are supressed by a factor of nine over unpolarized beams.  Although
the right-polarization will yield by far the most significant results,
we also consider the left-polarization, in which the W-pair background
is actually enhanced.

We select events with zero jets, two oppositely-charged tracks
identified as electrons or muons, and more than 350 GeV of missing
energy.  For a right-polarized beam, these are our only requirements
(Figure \ref{jimptwoleptons}).  For a left-polarized beam, swamped with
W-pair backgrounds, we also require that the polar angle of the
positive lepton with respect to the incident positron beam be less
than the polar angle of the negative lepton.  This avoids the region
in which the direction of the $W^\pm$ is correlated with the incident
$e^\pm$, reducing this background by more than a factor of ten.  The
neutralino signal, being forward-backward symmetric, is only reduced
by a factor of two (Figure \ref{jimpleftleptons}).

The primary backgrounds are W-pairs, Z-pairs, and light charginos.
W-pairs are either suppressed by right-polarization or the angular cut
described above, and Z-pairs have predictable missing energy outside
the regions of interest, leaving charginos as the only irreducible
background.  W-pairs and charginos can be subtracted without model
dependence by noting that signal neutralinos always decay to two
leptons of the same flavor, while W-pairs and charginos decay to two
uncorrelated lepton flavors.  (Each chargino decays to
$\tilde{\chi}^0_1$ with a $W^\pm$, which decays to a lepton and a
neutrino.)  If, therefore, we assume perfect flavor tagging and
subtract opposite-flavor events from the invariant mass plot, we
obtain a histogram with the two neutralino modes superimposed, W-pair
and chargino backgrounds only contributing to statistical error, and
all Z-pair events in one bin (91 GeV) above the maximum neutralino
invariant mass ($\sim$80 GeV).  These histograms are shown on the
rights of Figures \ref{jimptwoleptons} and \ref{jimpleftleptons}.

The combined cross-sections of $\tilde{\chi}^0_1\tilde{\chi}^0_3$ and
$\tilde{\chi}^0_2\tilde{\chi}^0_3$ can be determined by integrating
this histogram up to about 85 GeV (to exclude bias from the Z-pairs
and statistical error from the W-pair and chargino subtraction).  For
250 fb$^{-1}$, this yields 1093 $\pm$ 65 for a right-polarized beam
and 720 $\pm$ 91 for a left-polarized beam with the cuts described.
The coefficients for the weighted sum of cross-sections are derived
from branching fractions and efficiencies for each mode into the two
detected leptons with all applied cuts.  The branching fractions times
efficiencies are given in Table
\ref{jimpbetable}.

\begin{table}[!h]
  \begin{center}
    \renewcommand{\arraystretch}{1.5}
    \begin{tabular}{c c c}
      & electrons 95\% left-polarized & electrons 95\% right-polarized \\
      $\tilde{\chi}^0_1\tilde{\chi}^0_3$ & 0.0308 & 0.0584 \\
      $\tilde{\chi}^0_2\tilde{\chi}^0_3$ & 0.0123 & 0.0254
    \end{tabular}
    \renewcommand{\arraystretch}{1}
    \vspace{-0.5 cm}
  \end{center}

  \caption{Branching fractions times efficiencies for left- and
  right-polarized electron beams.  \label{jimpbetable}}

\end{table}

Dividing these by the number of events seen and multiplying by 250
fb$^{-1}$, we obtain the constraints on the left- and right-handed
cross-sections of $\tilde{\chi}^0_1\tilde{\chi}^0_3$ ($\sigma_{13}^L$
and $\sigma_{13}^R$) and $\tilde{\chi}^0_2\tilde{\chi}^0_3$
($\sigma_{23}^L$ and $\sigma_{23}^R$).
\begin{eqnarray}
  0.0107 \, \sigma_{13}^L + 0.00427 \, \sigma_{23}^L &=& 1 \pm 0.13 \, \sqrt{\frac{250 \mbox{ fb}^{-1}}{\mathcal{L}}} \\
  0.0133 \, \sigma_{13}^R + 0.00580 \, \sigma_{23}^R &=& 1 \pm 0.059 \, \sqrt{\frac{250 \mbox{ fb}^{-1}}{\mathcal{L}}}
\end{eqnarray}
where $\mathcal{L}$ is the total integrated luminosity.

Left- and right-polarized neutralino cross-sections are expected to
differ by about 30\%, but this difference will probably not be
observable.  Assuming a perfect $\sigma_{23}$ subtraction, the
difference between $\sigma_{13}^L$ (56.8 fb) and $\sigma_{13}^R$ (44.1
fb) can only be distinguished at the 3-$\sigma$ level with 900
fb$^{-1}$ of right-polarized collisions.

To measure the mass difference between $\tilde{\chi}^0_3$ and
$\tilde{\chi}^0_1$, we fit our lepton invariant mass spectrum to the
neutralino threshold function discussed in section \ref{Y}.  The
Z-pair bin contributes to the $\chi^2$ of the fit, but not any of its
derivatives near the minimum (since the fit function is exactly flat
at 91 GeV).  (It could just as easily be dropped from the fit.)  While
fits of this function to the histograms shown on the rights of Figures
\ref{jimptwoleptons} and \ref{jimpleftleptons} converged, they yielded
central values which depend linearly on the bin spacing for small
variations, even with five times as many bins!  For a different final
state (section \ref{Z}), this pathology is avoided by employing an
unbinned maximum likelihood fit, but such a technique would be hard to
implement without giving up the model-independent background
subtraction described above.  Since statistical errors should be
independent of the fitting technique, we report errors from our binned
$\chi^2$ minimization, which do not depend on bin spacing: $\pm$0.7
GeV from right-polarized electrons and $\pm$2 GeV from left-polarized
electrons.

\section{Light Charginos to Two Jets, One Lepton}

\begin{figure}[t]
  \begin{center}
    \begin{tabular}{p{0.49\linewidth} p{0.49\linewidth}}
      \begin{minipage}{\linewidth} \begin{center} Electrons left-polarized 95\% \end{center} \end{minipage} &
      \begin{minipage}{\linewidth} \begin{center} Electrons right-polarized 95\% \end{center} \end{minipage} \\
      \begin{minipage}{\linewidth} \includegraphics[width=\linewidth]{again4.pdf} \end{minipage} &
      \begin{minipage}{\linewidth} \includegraphics[width=\linewidth]{again3.pdf} \end{minipage}
    \end{tabular}

    \caption{Event selection for light charginos from 250 fb$^{-1}$ of
    $e^+e^-$ collisions: the left set of plots have electrons 95\%
    left-polarized and the right set have electrons 95\% right
    polarized; positrons are unpolarized in both.  Within each set,
    the top plot is the missing energy for events with two jets and
    one lepton.  Bottom-left: $\cos\theta$ distribution of each jet
    where $\theta$ is the polar angle, with $>$ 300 GeV missing
    energy.  Bottom-right: Invariant mass of the two jets with missing
    energy and $|\cos\theta| >$ 0.9.  SUSY charginos are distinguished
    from Standard Model W-pairs by requiring the invariant mass to be
    $<$ 60 GeV.}

    \label{jimpcharginocuts}
  \end{center}
\end{figure}

\begin{center}
  \begin{tabular}{l l l l l l l}
    $e^+e^- \to $ & $\tilde{\chi}^+_1$ & & & $\tilde{\chi}^-_1$ & & \\
    & $\tilde{\chi}^+_1$ & $\to \tilde{\chi}^0_1$ & $W^+$ & $\tilde{\chi}^-_1$ & $\to \tilde{\chi}^0_1$ & $W^-$ \\
    & & & $W^+ \to jj$ & & & $W^- \to \ell^- \bar{\nu}$ \\
  \end{tabular}
\end{center}

\bigskip

The light chargino mode ($\tilde{\chi}^+_1 \tilde{\chi}^-_1$) is the
most copious of the four SUSY decays studied, and can be very cleanly
separated from all backgrounds by requiring two jets, exactly one
isolated track, and missing energy.  Each $\tilde{\chi}^\pm_1$ decays
into a $\tilde{\chi}^0_1$ and a virtual $W^\pm$, one $W^\pm$ decays
into two jets and the other into an electron or muon, and a neutrino.
The two ground-state neutralinos disappear as missing energy (along
with the neutrino), and the single lepton tags the event as the decay
of a pair of charged particles through $W^\pm$.  The invariant mass of
the two jets may then be used to veto Standard Model on-shell W
bosons, and then measure the $\tilde{\chi}^\pm_1$, $\tilde{\chi}^0_1$
mass difference by identifying the upper threshold.  Since the initial
state energy is known and the charginos can be cleanly separated from
all backgrounds, the absolute mass of the LSP $\tilde{\chi}^0_1$
can be estimated from the kinematic envelope of the two jets.

Two polarization states were studied: one in which the electrons are
95\% left-polarized and another in which the electrons are 95\%
right-polarized, with unpolarized positrons in either case.  After
requiring two distinguishable jets and an isolated track, identified
as a lepton, we require events to have more than 300 GeV of missing
energy.  At this point, the only significant background is Standard
Model W-pairs.  We further reject events in which either jet points
within 25 degrees of the beamline ($|\cos\theta| >$ 0.9 for polar
angle $\theta$), so that all of the energy and momentum of each jet is
accounted for.  Then we plot the invariant mass of the two jets, and
reject the peak from on-shell W-pairs by requiring the invariant mass
to be less than 60 GeV.  All of these selection criteria are presented
in Figure \ref{jimpcharginocuts}.

With these cuts, the sum of all physics backgrounds constitutes about
5--8\% of the measured events, so even after background subtraction,
the uncertainty in the number of chargino events seen can be taken to
be approximately the square root of this number.  With $\sigma$,
$\mathcal{L}$, $\mathcal{B}$, and $\varepsilon$ as the light chargino
cross-section, integrated luminosity, branching fraction to our final
state, and the efficiency of the cuts, respectively, the number of
events after cuts ($\sigma\mathcal{L}\mathcal{B}\varepsilon$) has
uncertainty $\sqrt{\sigma\mathcal{L}\mathcal{B}\varepsilon}$.
Therefore, the cross-section has uncertainty
\begin{equation}
  \sigma \pm \sqrt{\frac{\sigma}{\mathcal{L}\mathcal{B}\varepsilon}}.
\end{equation}
The product $\mathcal{B}\varepsilon$ is 7.7\% for charginos produced
with left-handed electrons and 10.5\% for charginos produced with
right-handed electrons, so the uncertainty in these cross-section
measurements scale as
\begin{equation}
  \sigma_L = 940\mbox{ fb} \pm \sqrt{\frac{940\mbox{ fb}}{0.077 \, \mathcal{L}}}
  \mbox{\hspace{0.5 cm} and \hspace{0.5 cm}}
  \sigma_R = 119\mbox{ fb} \pm \sqrt{\frac{119\mbox{ fb}}{0.105 \, \mathcal{L}}}.
\end{equation}
With 250 fb$^{-1}$, $\sigma_L$ can be measured to $\pm$7.0 fb and
$\sigma_R$ can be measured to $\pm$2 fb.

\begin{figure}[t]
  \begin{center}
    \begin{tabular}{p{0.49\linewidth} p{0.49\linewidth}}
      \begin{minipage}{\linewidth} \includegraphics[width=\linewidth]{again5.pdf} \end{minipage} &
      \begin{minipage}{\linewidth} \includegraphics[width=\linewidth]{again6.pdf} \end{minipage}
    \end{tabular}
    \caption{Light charginos from 250 fb$^{-1}$ of $e^+e^-$ collisions
    with electrons 95\% left-polarized and positrons unpolarized.
    Left: Maximum liklihood fit to two-jet invariant mass above 25
    GeV.  (Pairs of jets with lower invariant masses are likely to
    overlap and have poor efficiency.)  Right: Invariant mass
    (vertical) versus total jet energy (horizontal) with W-pair
    backgrounds included.  Solid and dashed lines are kinematic
    envelope for a $\tilde{\chi}^0_1$ of 106.5 GeV, $\pm$5 GeV.  (The
    chargino-neutralino mass difference is kept fixed).}
    \label{jimpcharginoanal}
  \end{center}
\end{figure}

The upper threshold in the two jet distribution can be used to measure
the mass difference between $\tilde{\chi}^\pm_1$ and
$\tilde{\chi}^0_1$.  We proceed by fitting the left-polarized jet
invariant mass to the function described in section \ref{W}, which,
for our purposes, has two free parameters: the desired mass difference
and $\zeta = (|C_V|^2 - |C_A|^2)/(|C_V|^2 + |C_A|^2)$.  Anticipating a
sensitivity to bin spacing, we perform an unbinned maximum liklihood
fit to this function, with Gaussian smearing to adequately represent
mismeasured jets that lie above threshold.  A Gaussian smearing width
of 1.2 GeV is taken from the width of the real $W^\pm$ peak.
Invariant masses below 25 GeV are dropped from the fit, since pairs of
jets with low invariant masses are likely to overlap, leading to low
jet-finding efficiency.  From this fit, we obtain an uncertainty in
mass difference of $\pm$0.3 GeV and an $\zeta$ of 0.95 $\pm$0.02.
(See the left of Figure \ref{jimpcharginoanal}.)

While the invariant mass distribution alone is insensitive to the
absolute mass of $\tilde{\chi}^0_1$, this value can be read off the
envelope of the invariant mass versus total jet energy distribution.
The right of Figure \ref{jimpcharginoanal} shows this two-dimensional
distribution with the W-pair background included (for scale).  The
kinematic outer edge of this distribution is given by
\begin{equation}
  m_jj = \sqrt{\mbox{$E_{jj}$}^2-\left(\sqrt{\frac{E_{CM}}{2}^2-m_{\tilde{\chi}^\pm_1}^2}-\sqrt{\left(\frac{E_{CM}}{2}-E_{jj}\right)^2-m_{\tilde{\chi}^0_1}^2}\right)^2}
\end{equation}
where $m_jj$ is the two-jet invariant mass, $E_{jj}$ is the two-jet
energy, $m_{\tilde{\chi}^\pm_1}$ is the light chargino mass, and
$m_{\tilde{\chi}^0_1}$ is the lightest stable particle mass.  This
boundary is drawn on Figure \ref{jimpcharginoanal} in solid red, with
$\pm$5 GeV variations in both $m_{\tilde{\chi}^0_1}$ and
$m_{\tilde{\chi}^\pm_1}$ as dashed curves.  (The two masses are raised
and lowered 5 GeV, keeping the mass difference fixed.)  We can
estimate, therefore, that a more thorough analysis of this
distribution (including smearing of the two jets), will have on the
order of $\pm$5--10 GeV sensitivity to the mass of the lightest stable
particle, and therefore the absolute scale of the SUSY mass spectrum.

\end{document}
