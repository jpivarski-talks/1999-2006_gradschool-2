\documentclass[landscape]{article}
\usepackage[pdftex]{graphicx}
\pagestyle{empty}
\oddsidemargin  -0.5 in
\evensidemargin -0.5 in
\headheight     0 in
\topmargin      -1 in
\textheight     7.7 in
\textwidth      10 in
\begin{document}
\huge
\renewcommand{\labelitemi}{-}
\setlength{\parindent}{0 cm}

\begin{tabular}{p{0.45\linewidth} p{0.25\linewidth} p{0.25\linewidth}}
  \begin{center} $\Upsilon(1S)$ \end{center} & \begin{center} $\Upsilon(2S)$ \end{center} & \begin{center} $\Upsilon(3S)$ \end{center} \\
  \begin{minipage}{\linewidth}
    \includegraphics[width=7.7 cm, angle=-90, viewport=0 0 316 475, clip=true]{../tables5/all1_bblumi_normcc.pdf}
  \end{minipage} &
  \begin{minipage}{\linewidth}
    \includegraphics[width=7.7 cm, angle=-90, viewport=0 210 316 475, clip=true]{../tables5/all2_bblumi_normcc.pdf}
  \end{minipage} &
  \begin{minipage}{\linewidth}
    \includegraphics[width=7.7 cm, angle=-90, viewport=0 210 316 475, clip=true]{../tables5/all3_bblumi_normcc.pdf}
  \end{minipage} \\
\end{tabular}

\vfill

Monte Carlo is hadronic (no prompt leptons, but everything else
(including {\it cascades} to leptons)).

\vfill

Data is hadronic (on-res minus off-res minus beamgas (from
single-beam) minus cosmic rays (from no-beam) minus prompt decays to
$e^+e^-$, $\mu^+\mu^-$, $\tau^+\tau^-$ (from Monte Carlo)).

\vfill

\pagebreak

Strategy for determining efficiency

\vfill

\begin{tabular}{p{0.35\linewidth} p{0.6\linewidth}}
  \begin{minipage}{\linewidth}
    $\left\{ \begin{array}{c}
      \vspace{0.5 cm} \mbox{trigger} \\
      \vspace{0.5 cm} \mbox{closest d0 $<$ 5 mm} \\
      \mbox{biggest shower $<$} \\
      \vspace{0.5 cm} \mbox{85\% beam energy} \\
      \mbox{second-biggest track $<$} \\
      \mbox{85\% beam energy}
    \end{array} \right\}$
  \end{minipage} &
  \begin{minipage}{\linewidth}
    Background subtraction dominates uncertainties

    \vspace{1 cm}
    Data/MC agreement looks good

    \vspace{1 cm}
    Cut boundaries are far from signal

    \vspace{1 cm}
    Measure from Monte Carlo
  \end{minipage}
\end{tabular}

\vfill

\begin{tabular}{p{0.35\linewidth} p{0.6\linewidth}}
  \begin{minipage}{\linewidth}
    $\left\{ \begin{array}{c}
      \mbox{Event vertex Z $<$ 7.5 cm} \\
    \end{array} \right\}$
  \end{minipage} &
  \begin{minipage}{\linewidth}
    Only $\Upsilon(1S)$ has sub-percent errors

    Measure from $\Upsilon(1S)$, apply to all three
  \end{minipage}
\end{tabular}

\vfill

\begin{tabular}{p{0.35\linewidth} p{0.6\linewidth}}
  \begin{minipage}{\linewidth}
    $\left\{ \begin{array}{c}
      \mbox{visible energy $<$ 35\% of } \\
      \vspace{0.5 cm} \mbox{center-of-mass energy} \\
      \vspace{0.5 cm} \mbox{quality tracks $\ge$ 2} \\
      \vspace{0.5 cm} \mbox{level 3/level 4} \\
      \mbox{CC energy $<$ 85\% of} \\
      \mbox{center-of-mass energy} \\
    \end{array} \right\}$
  \end{minipage} &
  \begin{minipage}{\linewidth}
    No more backgrounds

    \vspace{1 cm}
    Measure from data
  \end{minipage} \\

\end{tabular}

\pagebreak

Trigger variables

\vspace{-1.75 cm}
\begin{center}
  \begin{tabular}{p{0.32\linewidth} p{0.32\linewidth} p{0.32\linewidth}}
    \begin{center} $\Upsilon(1S)$ \end{center} & \begin{center} $\Upsilon(2S)$ \end{center} & \begin{center} $\Upsilon(3S)$ \end{center} \\
  \end{tabular}

  \vspace{-0.75 cm}
  \begin{tabular}{p{0.32\linewidth} p{0.32\linewidth} p{0.32\linewidth}}
    \includegraphics[width=\linewidth]{../tables5/all1_bblumi_normcc_trigger.pdf} &
    \includegraphics[width=\linewidth]{../tables5/all2_bblumi_normcc_trigger.pdf} &
    \includegraphics[width=\linewidth]{../tables5/all3_bblumi_normcc_trigger.pdf}
  \end{tabular}
\end{center}

\begin{itemize}
  \item Blue histogram is Monte Carlo

  \item Bluish bins are prompt decays to $e^+e^-$ and $\mu^+\mu^-$

  \item Greenish bins are prompt decays to $\tau^+\tau^-$

  \item Red data points are the sum of these + beam-gas and cosmic rays

  \item Black points are on-res minus off-res data

  \item Black points should agree with red points
\end{itemize}

\vfill

Suppose we take the error on trigger to be 100\% of itself: 0.5\% efficiency systematic.

\pagebreak

d0 of closest track $<$ 5 mm (trigger already implied existence of one track)

\vspace{-1.75 cm}
\begin{center}
  \begin{tabular}{p{0.32\linewidth} p{0.32\linewidth} p{0.32\linewidth}}
    \begin{center} $\Upsilon(1S)$ \end{center} & \begin{center} $\Upsilon(2S)$ \end{center} & \begin{center} $\Upsilon(3S)$ \end{center} \\
  \end{tabular}

  \vspace{-0.75 cm}
  \begin{tabular}{p{0.32\linewidth} p{0.32\linewidth} p{0.32\linewidth}}
    \includegraphics[width=\linewidth]{../tables5/all1_bblumi_normcc_d0close.pdf} &
    \includegraphics[width=\linewidth]{../tables5/all2_bblumi_normcc_d0close.pdf} &
    \includegraphics[width=\linewidth]{../tables5/all3_bblumi_normcc_d0close.pdf}
  \end{tabular}
\end{center}

(Fixed beamspot problem!)

\vfill

Taking error to be 100\% of cut region would yield a 0.1\% efficiency systematic

\vfill

Suppose we're wrong the other way: shift cut from 5 mm down to 2 mm
changes the result by 0.25\%

\vfill

Take efficiency systematic to be 0.25\%.

\vfill

\pagebreak

Biggest shower $<$ 0.85\% of beam energy

\vspace{-1.75 cm}
\begin{center}
  \begin{tabular}{p{0.32\linewidth} p{0.32\linewidth} p{0.32\linewidth}}
    \begin{center} $\Upsilon(1S)$ \end{center} & \begin{center} $\Upsilon(2S)$ \end{center} & \begin{center} $\Upsilon(3S)$ \end{center} \\
  \end{tabular}

  \vspace{-0.75 cm}
  \begin{tabular}{p{0.32\linewidth} p{0.32\linewidth} p{0.32\linewidth}}
    \includegraphics[width=\linewidth]{../tables5/all1_bblumi_normcc_e1.pdf} &
    \includegraphics[width=\linewidth]{../tables5/all2_bblumi_normcc_e1.pdf} &
    \includegraphics[width=\linewidth]{../tables5/all3_bblumi_normcc_e1.pdf}
  \end{tabular}
\end{center}

\vfill

$ggg$ events are all easily below the cut-off

\vfill

$gg\gamma$ span the boundary

\vfill

cascade decays to $e^+e^-$ are all to the right of the boundary

\vfill

\pagebreak

Biggest shower uncertainty from $\Gamma_{gg\gamma}/\Gamma_{ggg}$

\begin{tabular}{p{0.5\linewidth} p{0.45\linewidth}}
  \begin{minipage}{\linewidth}
    \includegraphics[width=\linewidth]{../ggg_vs_gggamma.pdf}
  \end{minipage} &
  \begin{minipage}{\linewidth}
    Solid is $ggg$, dashed is $gg\gamma$ (boundary is also dashed)

    \vspace{1 cm}
    $\Gamma_{gg\gamma}/\Gamma_{ggg}$ is precise
    \begin{itemize}
      \item 2.75 $\pm$ 0.15\% from CLEO PRD 55, 5273 direct measurement

      \item 3.646 $\pm$ 0.054\% from PDG world-average and running to $M_\Upsilon$
    \end{itemize}

    \vspace{1 cm}
    Even if we straddle both, efficiency gets a systematic of 0.15\% error

  \end{minipage}
\end{tabular}

\vfill

\begin{center} But wait!  There's a disaster! \end{center}

\vfill

\pagebreak

Disasterous uncertainty in $gg\gamma$ Monte Carlo

\begin{tabular}{p{0.5\linewidth} p{0.45\linewidth}}
  \includegraphics[width=\linewidth]{../ggg_vs_gggamma_again.pdf} &
  \includegraphics[width=\linewidth]{../example_big_gggamma.pdf}
\end{tabular}

\vfill

\begin{flushright} (Yes, that says 28.4 GeV in the bottom shower) \end{flushright}

\vfill

The big showers don't come from the prompt photon, they come from a
massively overlapped jet from the $gg$.  This tail extends way beyond
the full center-of-mass energy.

\vfill

\begin{center} \fbox{Is this feature in the data?} \end{center}

\pagebreak

No.  Not even a little bit.

\vspace{-1.75 cm}
\begin{center}
  \begin{tabular}{p{0.32\linewidth} p{0.32\linewidth} p{0.32\linewidth}}
    \begin{center} $\Upsilon(1S)$ \end{center} & \begin{center} $\Upsilon(2S)$ \end{center} & \begin{center} $\Upsilon(3S)$ \end{center} \\
  \end{tabular}

  \vspace{-0.75 cm}
  \begin{tabular}{p{0.32\linewidth} p{0.32\linewidth} p{0.32\linewidth}}
    \includegraphics[width=\linewidth]{../mc_overmodelling_shower_pileup.pdf} &
    \includegraphics[width=\linewidth]{../mc_overmodelling_shower_pileup2.pdf} &
    \includegraphics[width=\linewidth]{../mc_overmodelling_shower_pileup3.pdf}
  \end{tabular}
\end{center}

\begin{itemize}
  \item Horizontal axis is in units of center-of-mass energy

  \item Solid is Monte Carlo

  \item Dashed is on-res minus off-res

  \item Histogram is just on-res
\end{itemize}

\vfill

If Monte Carlo didn't make this mistake, would $gg\gamma$ events be
below the cut boundary?  I can't tell, so I'll take uncertainty in
$\Gamma_{gg\gamma}/\Gamma_{ggg}$ to be all of itself, incurring a
1.1\% systematic in efficiency.

\vfill

\pagebreak

Second-biggest track $<$ 0.85\% of beam energy

\vspace{-1.75 cm}
\begin{center}
  \begin{tabular}{p{0.32\linewidth} p{0.32\linewidth} p{0.32\linewidth}}
    \begin{center} $\Upsilon(1S)$ \end{center} & \begin{center} $\Upsilon(2S)$ \end{center} & \begin{center} $\Upsilon(3S)$ \end{center} \\
  \end{tabular}

  \vspace{-0.75 cm}
  \begin{tabular}{p{0.32\linewidth} p{0.32\linewidth} p{0.32\linewidth}}
    \includegraphics[width=\linewidth]{../tables5/all1_bblumi_normcc_p2.pdf} &
    \includegraphics[width=\linewidth]{../tables5/all2_bblumi_normcc_p2.pdf} &
    \includegraphics[width=\linewidth]{../tables5/all3_bblumi_normcc_p2.pdf}
  \end{tabular}
\end{center}

\vfill

No such problems here.  All $ggg$ and $gg\gamma$ pass this cut.

\vfill

I haven't forgotten the cascade to leptons systematic.

\vspace{0.5 cm}
With Istvan's new $\mathcal{B}_{\mu\mu}$, $\Upsilon(nS) \to X \ell^+
\ell^-$ (where $X$ cannot be nothing)
\begin{center}
is 0 for $\Upsilon(1S)$, 0.795 $\pm$ 0.036\% for
$\Upsilon(2S)$ and 0.480 $\pm$ 0.028\% for $\Upsilon(3S)$.
\end{center}

\vfill

Systematic error on efficiency for these uncertainties is 0.06\%.

\vfill

Supposing PHOTOS is wrong by 50\%, we get another systematic of
0.03\%.

\pagebreak

Event vertex Z $<$ 7.5 cm (or closest track z0 $<$ 7.5 cm, if only one track)

\vspace{-1.75 cm}
\begin{center}
  \begin{tabular}{p{0.32\linewidth} p{0.32\linewidth} p{0.32\linewidth}}
    \begin{center} $\Upsilon(1S)$ \end{center} & \begin{center} $\Upsilon(2S)$ \end{center} & \begin{center} $\Upsilon(3S)$ \end{center} \\
  \end{tabular}

  \vspace{-0.75 cm}
  \begin{tabular}{p{0.32\linewidth} p{0.32\linewidth} p{0.32\linewidth}}
    \includegraphics[width=\linewidth]{../tables5/all1_bblumi_normcc_wz.pdf} &
    \includegraphics[width=\linewidth]{../tables5/all2_bblumi_normcc_wz.pdf} &
    \includegraphics[width=\linewidth]{../tables5/all3_bblumi_normcc_wz.pdf}
  \end{tabular}
\end{center}

(Bottom plot is closest z0 for those few events which didn't have two
tracks to form a Z vertex.  It doesn't cancel well because I forgot to
move z0 to the beamspot--- $\Upsilon(2S)$ and $\Upsilon(3S)$ samples
contain many runs.)

\vfill

This cut is perhaps a little too close to trust Monte Carlo,
backgrounds are now controlled for $\Upsilon(1S)$ (the tallest peak),
and cut efficiency should be the same for all three resonances, so
take $\Upsilon(1S)$ value for each.

\vspace{-1.5 cm}
\begin{center}
  \begin{tabular}{p{0.32\linewidth} p{0.32\linewidth} p{0.32\linewidth}}
    \begin{center} 99.35 $\pm$ 0.20 $\pm$ 0.56\% \end{center} &
    \begin{center} 99.96 $\pm$ 0.30 $\pm$ 1.39\% \end{center} &
    \begin{center} 99.53 $\pm$ 0.41 $\pm$ 1.99\% \end{center}
  \end{tabular}
\end{center}

\vspace{-0.75 cm}
First error is statistical (binomial fluctuations in samples that pass
cuts) and second is systematic (fluctuations in control samples, scale factors).

\pagebreak

Visible energy $<$ 0.35\% of center-of-mass energy

\vspace{-1.75 cm}
\begin{center}
  \begin{tabular}{p{0.32\linewidth} p{0.32\linewidth} p{0.32\linewidth}}
    \begin{center} $\Upsilon(1S)$ \end{center} & \begin{center} $\Upsilon(2S)$ \end{center} & \begin{center} $\Upsilon(3S)$ \end{center} \\
  \end{tabular}

  \vspace{-0.75 cm}
  \begin{tabular}{p{0.32\linewidth} p{0.32\linewidth} p{0.32\linewidth}}
    \includegraphics[width=\linewidth]{../tables5/all1_bblumi_normcc_visen.pdf} &
    \includegraphics[width=\linewidth]{../tables5/all2_bblumi_normcc_visen.pdf} &
    \includegraphics[width=\linewidth]{../tables5/all3_bblumi_normcc_visen.pdf}
  \end{tabular}
\end{center}

\vfill

Bottom plot is the on-res (blue) and off-res (red) before subtraction.

\vfill

The low-energy bump seems to be mostly two-photon events.

\vfill

The problem: $\Upsilon(3S)$ doesn't subtract perfectly--- low-energy
on-res events got about 50 MeV more neutral energy per event than
off-res events.

\vfill

\pagebreak

The solution:

\vspace{0.5 cm}
\begin{tabular}{p{0.45\linewidth} p{0.45\linewidth}}
  \includegraphics[width=0.9\linewidth]{../tables5/all3_bblumi_normcc_visen.pdf} &
  \includegraphics[width=0.9\linewidth]{../tables5/all3_bblumi_coolcc_visen.pdf}
\end{tabular}

\begin{tabular}{p{0.45\linewidth} p{0.45\linewidth}}
  \begin{minipage}{\linewidth} \begin{center} Normal visible energy \end{center} \end{minipage} &
  \begin{minipage}{\linewidth} \begin{center} No hot showers \end{center} \end{minipage}
\end{tabular}

\vfill

Calculate visible energy excluding hot showers and the mismatch is gone.

\vfill

Tau subcollection cuts on visible energy including hot showers.

\vfill

I will continue to cut on that variable, but raise the threshold
above the dangerous bump.

\vfill

\pagebreak

Quality tracks $\ge$ 2

\vspace{-1.75 cm}
\begin{center}
  \begin{tabular}{p{0.32\linewidth} p{0.32\linewidth} p{0.32\linewidth}}
    \begin{center} $\Upsilon(1S)$ \end{center} & \begin{center} $\Upsilon(2S)$ \end{center} & \begin{center} $\Upsilon(3S)$ \end{center} \\
  \end{tabular}

  \vspace{-0.75 cm}
  \begin{tabular}{p{0.32\linewidth} p{0.32\linewidth} p{0.32\linewidth}}
    \includegraphics[width=\linewidth]{../tables5/all1_bblumi_normcc_tracks.pdf} &
    \includegraphics[width=\linewidth]{../tables5/all2_bblumi_normcc_tracks.pdf} &
    \includegraphics[width=\linewidth]{../tables5/all3_bblumi_normcc_tracks.pdf}
  \end{tabular}
\end{center}

\vfill

This was the reason I switched to the tau subcollection: data/MC
mismatch doesn't matter when the cut is made so low.

\vfill

Nor does it matter now anyway, since I'm reading the efficiency from
data for this variable.

\vfill

\pagebreak

CC energy $<$ 0.85\% of center-of-mass energy

\vspace{-1.75 cm}
\begin{center}
  \begin{tabular}{p{0.32\linewidth} p{0.32\linewidth} p{0.32\linewidth}}
    \begin{center} $\Upsilon(1S)$ \end{center} & \begin{center} $\Upsilon(2S)$ \end{center} & \begin{center} $\Upsilon(3S)$ \end{center} \\
  \end{tabular}

  \vspace{-0.75 cm}
  \begin{tabular}{p{0.32\linewidth} p{0.32\linewidth} p{0.32\linewidth}}
    \includegraphics[width=\linewidth]{../tables5/all1_bblumi_normcc_ccen.pdf} &
    \includegraphics[width=\linewidth]{../tables5/all2_bblumi_normcc_ccen.pdf} &
    \includegraphics[width=\linewidth]{../tables5/all3_bblumi_normcc_ccen.pdf}
  \end{tabular}
\end{center}

\vfill

This was the data/MC disagreement I introduced by the switch.  I will
remind you that I'm measuring the efficiency from the data.

\vfill

\pagebreak

The big table

\vfill

\begin{center}\fbox{\begin{minipage}{0.9\linewidth}
\begin{center}\begin{tabular}{l l l l}
  & \mbox{\hspace{0.55 cm}} $\Upsilon(1S)$ \mbox{\hspace{1 cm}}
  & \mbox{\hspace{0.55 cm}} $\Upsilon(2S)$ \mbox{\hspace{1 cm}}
  & \mbox{\hspace{0.55 cm}} $\Upsilon(3S)$ \mbox{\hspace{1 cm}} \\\hline\hline
  Monte Carlo part \mbox{\hspace{4 cm}} & \mbox{\hspace{0.55 cm}} 0.9852 & \mbox{\hspace{0.55 cm}} 0.9711 & \mbox{\hspace{0.55 cm}} 0.9773 \\
  statistical & $\pm$ 0.00032 & $\pm$ 0.00040 & $\pm$ 0.00048 \\
  trigger & $\pm$ 0.0050 & $\pm$ 0.0050 & $\pm$ 0.0050 \\
  closest d0 & $\pm$ 0.0025 & $\pm$ 0.0025 & $\pm$ 0.0025 \\
  100\% of $\Gamma_{gg\gamma}/\Gamma_{ggg}$ & $\pm$ 0.0111 & $\pm$ 0.0108 & $\pm$ 0.0109 \\
  $\mathcal{B}$ of cascade-leptons (1$\sigma$) & $\pm$ 0 & $\pm$ 0.0006 & $\pm$ 0.0005 \\
  PHOTOS by 50\% & $\pm$ 0 & $\pm$ 0.0003 & $\pm$ 0.0001 \\\hline
  event Z & \mbox{\hspace{0.55 cm}} 0.9935 & \mbox{\hspace{0.55 cm}} 0.9935 & \mbox{\hspace{0.55 cm}} 0.9935 \\
  statistical & $\pm$ 0.0020 & $\pm$ 0.0020 & $\pm$ 0.0020 \\
  systematic & $\pm$ 0.0056 & $\pm$ 0.0056 & $\pm$ 0.0056 \\\hline
  data part & \mbox{\hspace{0.55 cm}} 0.9949 & \mbox{\hspace{0.55 cm}} 0.9792 & \mbox{\hspace{0.55 cm}} 0.9874 \\
  statistical & $\pm$ 0.0034 & $\pm$ 0.0039 & $\pm$ 0.0055 \\
  systematic & $\pm$ 0.0081 & $\pm$ 0.0088 & $\pm$ 0.0110 \\\hline\hline
  & \mbox{\hspace{0.55 cm}} 0.9738 & \mbox{\hspace{0.55 cm}} 0.9447 & \mbox{\hspace{0.55 cm}} 0.9587 \\
  total & $\pm$ 0.0040 & $\pm$ 0.0044 & $\pm$ 0.0059 \\
  & $\pm$ 0.0159 & $\pm$ 0.0160 & $\pm$ 0.0174
\end{tabular}\end{center}\end{minipage}}\end{center}

\vfill

\mbox{ }

\vfill

\end{document}
